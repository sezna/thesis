%
% spacing could probably be improved
%

\begin{center}

\bigskip

\begin{Large}
\textbf{\theTitle}
\end{Large}

\bigskip

\begin{large}
\theAuthor
\end{large}

\bigskip
\bigskip

\textbf{Abstract}

\end{center}

\noindent
This thesis contains two projects. A modified Huffman code is presented as a lossless method to compress common traffic types. We posit the usage of compression benefits instead of just frequency of occurrence, as is common in Huffman codes, as the priority of each node when constructing the Huffman tree. We show the effectiveness of this method on common data transmission types and describe what would be needed for adoption of this new algorithm. 

We explore genetic algorithms as a method to create paintings based on images. We find a balance between computational work required and visually pleasing results to the algorithm, prioritizing aspects of the parameter space based on their impact on the painting and how they impact computational workload. 

